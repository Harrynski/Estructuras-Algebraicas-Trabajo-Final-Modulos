\documentclass[notes=show]{beamer}%
\usepackage{mathpazo}
\usepackage{hyperref}
\usepackage{multimedia}
\usepackage{amsmath}
\usepackage{amsfonts}
\usepackage{amssymb}
\usepackage{graphicx}%
\setcounter{MaxMatrixCols}{30}
%TCIDATA{OutputFilter=latex2.dll}
%TCIDATA{Version=5.50.0.2953}
%TCIDATA{CSTFile=beamer.cst}
%TCIDATA{Created=Thursday, July 01, 2021 12:41:31}
%TCIDATA{LastRevised=Sunday, July 31, 2022 20:20:43}
%TCIDATA{<META NAME="GraphicsSave" CONTENT="32">}
%TCIDATA{<META NAME="SaveForMode" CONTENT="1">}
%TCIDATA{BibliographyScheme=Manual}
%TCIDATA{<META NAME="DocumentShell" CONTENT="Other Documents\SW\Slides - Beamer">}
%TCIDATA{Language=American English}
%BeginMSIPreambleData
\providecommand{\U}[1]{\protect\rule{.1in}{.1in}}
%EndMSIPreambleData
\newenvironment{Defi}{\begin{block}{Definici\'{o}n}}{\end{block}}
\newenvironment{Lema}{\begin{block}{Lema}}{\end{block}}
\newenvironment{Corol}{\begin{block}{Corolario}}{\end{block}}
\newenvironment{Propo}{\begin{block}{Proposici\'{o}n}}{\end{block}}
\newenvironment{Teo}{\begin{block}{Teorema}}{\end{block}}
\newenvironment{Ej}{\begin{block}{Ejemplo}}{\end{block}}
\newenvironment{Ob}{\begin{block}{Observaci�n}}{\end{block}}
\newenvironment{stepenumerate}{\begin{enumerate}[<+->]}{\end{enumerate}}
\newenvironment{stepitemize}{\begin{itemize}[<+->]}{\end{itemize} }
\newenvironment{stepenumeratewithalert}{\begin{enumerate}[<+-| alert@+>]}{\end{enumerate}}
\newenvironment{stepitemizewithalert}{\begin{itemize}[<+-| alert@+>]}{\end{itemize} }
\newenvironment{ObV}{\begin{exampleblock}{Observaci�n}}{\end{exampleblock}}
\newenvironment{PropoR}{\begin{alertblock}{Proposici\'{o}n}}{\end{alertblock}}
\newenvironment{EjV}{\begin{exampleblock}{Ejemplo}}{\end{exampleblock}}
\usetheme{Madrid}
\begin{document}

\title{Intersecci\'{o}n, suma y base de m\'{o}dulos}
\subtitle{Estructuras Algebraicas}
\author{Tom\'{a}s Avila}
\institute{Universidad Nacional del Comahue}
\date{}
\maketitle

\bigskip%

%TCIMACRO{\TeXButton{BeginFrame}{\begin{frame}}}%
%BeginExpansion
\begin{frame}%
%EndExpansion


\bigskip%
%TCIMACRO{\QTR{frametitle}{M\'{o}dulo generado}}%
%BeginExpansion
\frametitle{M\'{o}dulo generado}%
%EndExpansion




\begin{Lema}
Sea $\Gamma$ una familia de subm\'{o}dulos del m\'{o}dulo $M$, entonces,%

\[%
%TCIMACRO{\dbigcap \limits_{A\in\Gamma}}%
%BeginExpansion
{\displaystyle\bigcap\limits_{A\in\Gamma}}
%EndExpansion
A=\{m\in M\,\ /\text{ }\forall A\in\Gamma\text{ }:\text{ }m\in A\},
\]


es un subm\'{o}dulo de $M$.
\end{Lema}

\begin{Corol}
$%
%TCIMACRO{\dbigcap \limits_{A\in\Gamma}}%
%BeginExpansion
{\displaystyle\bigcap\limits_{A\in\Gamma}}
%EndExpansion
A$ es el mayor subm\'{o}dulo que contiene a todos los elementos de la familia
$\Gamma$.
\end{Corol}

%

%TCIMACRO{\TeXButton{Transition: Box Out}{\transboxout}}%
%BeginExpansion
\transboxout
%EndExpansion%
%TCIMACRO{\TeXButton{EndFrame}{\end{frame}}}%
%BeginExpansion
\end{frame}%
%EndExpansion


\bigskip%

%TCIMACRO{\TeXButton{BeginFrame}{\begin{frame}}}%
%BeginExpansion
\begin{frame}%
%EndExpansion


\bigskip%
%TCIMACRO{\QTR{frametitle}{M\'{o}dulo generado}}%
%BeginExpansion
\frametitle{M\'{o}dulo generado}%
%EndExpansion


\begin{Lema}
Sea $X$ un subconjunto del m\'{o}dulo$\ M_{R}$, luego

$A=\left\{
\begin{array}
[c]{l}%
\{%
%TCIMACRO{\dsum \limits_{j=1}^{n}}%
%BeginExpansion
{\displaystyle\sum\limits_{j=1}^{n}}
%EndExpansion
x_{j}r_{j}\text{ }/\text{ }x_{j}\in X,\text{ }r_{j}\in R\text{, }n\in%
%TCIMACRO{\U{2115} }%
%BeginExpansion
\mathbb{N}
%EndExpansion
\},\text{ si }X\neq\varnothing\\
\{0\},\text{ si }X=\varnothing
\end{array}
\right.  ,$

es un subm\'{o}dulo de $M$.
\end{Lema}

\textbf{Demostraci\'{o}n:}

Si $X=\varnothing,$ es trivial.

Si $X\neq\varnothing,$ sean $a_{1},a_{2}\in A,$ as\'{\i} $a_{1}=%
%TCIMACRO{\dsum \limits_{i=1}^{m}}%
%BeginExpansion
{\displaystyle\sum\limits_{i=1}^{m}}
%EndExpansion
x_{i}r_{i}$, $a_{1}=%
%TCIMACRO{\dsum \limits_{j=1}^{n}}%
%BeginExpansion
{\displaystyle\sum\limits_{j=1}^{n}}
%EndExpansion
x_{j}r_{j}$, luego $a_{1}+a_{2}\in A.$ Adem\'{a}s, sea $r\in R$

$a_{1}r=(%
%TCIMACRO{\dsum \limits_{i=1}^{m}}%
%BeginExpansion
{\displaystyle\sum\limits_{i=1}^{m}}
%EndExpansion
x_{i}r_{i})r=%
%TCIMACRO{\dsum \limits_{i=1}^{m}}%
%BeginExpansion
{\displaystyle\sum\limits_{i=1}^{m}}
%EndExpansion
(x_{i}r_{i})r=%
%TCIMACRO{\dsum \limits_{i=1}^{m}}%
%BeginExpansion
{\displaystyle\sum\limits_{i=1}^{m}}
%EndExpansion
x_{i}r^{\prime}$

\bigskip

por lo tanto $a_{1}r\in A$. As\'{\i} $A$ es un subm\'{o}dulo de $M$.\bigskip

\bigskip%

%TCIMACRO{\TeXButton{Transition: Box Out}{\transboxout}}%
%BeginExpansion
\transboxout
%EndExpansion%
%TCIMACRO{\TeXButton{EndFrame}{\end{frame}}}%
%BeginExpansion
\end{frame}%
%EndExpansion


\bigskip%

%TCIMACRO{\TeXButton{BeginFrame}{\begin{frame}}}%
%BeginExpansion
\begin{frame}%
%EndExpansion


\bigskip%
%TCIMACRO{\QTR{frametitle}{M\'{o}dulo generado}}%
%BeginExpansion
\frametitle{M\'{o}dulo generado}%
%EndExpansion


\begin{Defi}
\bigskip El conjunto $A$ definido en el lema anterior es llamado
\textbf{subm\'{o}dulo a derecha generado }por $X$. Notamos $|X)$.
\end{Defi}

Este subm\'{o}dulo generado contiene a todas las combinaciones lineales
finitas $\sum x_{i}r_{i}$ y lo podemos caracterizar con el siguente lema.

\begin{Lema}
$|X)$ es el menor subm\'{o}dulo de $M$ que contiene a $X$, es decir,%

\[
|X)=%
%TCIMACRO{\dbigcap \limits_{\substack{C\hookrightarrow M\\X\subseteq C}}}%
%BeginExpansion
{\displaystyle\bigcap\limits_{\substack{C\hookrightarrow M\\X\subseteq C}}}
%EndExpansion
C\text{.}%
\]

\end{Lema}

\bigskip%

%TCIMACRO{\TeXButton{Transition: Box Out}{\transboxout}}%
%BeginExpansion
\transboxout
%EndExpansion%
%TCIMACRO{\TeXButton{EndFrame}{\end{frame}}}%
%BeginExpansion
\end{frame}%
%EndExpansion


\bigskip

\bigskip%
%TCIMACRO{\TeXButton{BeginFrame}{\begin{frame}}}%
%BeginExpansion
\begin{frame}%
%EndExpansion


\bigskip%
%TCIMACRO{\QTR{frametitle}{M\'{o}dulo generado}}%
%BeginExpansion
\frametitle{M\'{o}dulo generado}%
%EndExpansion


\bigskip\textbf{Dem. }Si $X=\varnothing,$ es trivial, pues $|X)=\{0\}$ y $X$
contiene al subm\'{o}dulo trivial $\{0\}$.

Supongamos que $X\neq\varnothing$ y sea $C$ un subm\'{o}dulo de $M$ que
contiene a $X$, luego $x_{i},$ $x_{i}r_{i}$ y todas las sumas finitas de estos
elementos estan en $C,$ sigue que $|X)\hookrightarrow C$.

Como $X\subseteq|X)$, ya que $x=x1,$ $\forall x\in X,$ tenemos que $|X)$ es
uno de los subm\'{o}dulos de $M$ que contienen a $X$, entonces $%
%TCIMACRO{\dbigcap \limits_{\substack{C\hookrightarrow M\\X\subseteq C}}}%
%BeginExpansion
{\displaystyle\bigcap\limits_{\substack{C\hookrightarrow M\\X\subseteq C}}}
%EndExpansion
C\subseteq|X)$.

\bigskip

\bigskip%

%TCIMACRO{\TeXButton{Transition: Box Out}{\transboxout}}%
%BeginExpansion
\transboxout
%EndExpansion%
%TCIMACRO{\TeXButton{EndFrame}{\end{frame}}}%
%BeginExpansion
\end{frame}%
%EndExpansion


\bigskip

%

%TCIMACRO{\TeXButton{BeginFrame}{\begin{frame}}}%
%BeginExpansion
\begin{frame}%
%EndExpansion


\bigskip%
%TCIMACRO{\QTR{frametitle}{M\'{o}dulo generado}}%
%BeginExpansion
\frametitle{M\'{o}dulo generado}%
%EndExpansion


\begin{Ob}
\bigskip\ 

\begin{itemize}
\item Sean $M$ un $R$-m\'{o}dulo a izquierda, $X\subseteq M$ y $X\neq
\varnothing$ entonces%

\[
(X|=\{%
%TCIMACRO{\dsum \limits_{j=1}^{n}}%
%BeginExpansion
{\displaystyle\sum\limits_{j=1}^{n}}
%EndExpansion
r_{j}x_{j}\text{ }/\text{ }x_{j}\in X,\text{ }r_{j}\in R\text{, }n\in%
%TCIMACRO{\U{2115} }%
%BeginExpansion
\mathbb{N}
%EndExpansion
\}
\]


\item Sean $M$ un S$R$-m\'{o}dulo, $X\subseteq M$ y $X\neq\varnothing$ entonces%

\[
(X)=\{%
%TCIMACRO{\dsum \limits_{j=1}^{n}}%
%BeginExpansion
{\displaystyle\sum\limits_{j=1}^{n}}
%EndExpansion
s_{j}x_{j}r_{j}\text{ }/\text{ }x_{j}\in X,\text{ }r_{j}\in R\text{, }s_{j}\in
S\text{, }n\in%
%TCIMACRO{\U{2115} }%
%BeginExpansion
\mathbb{N}
%EndExpansion
\}
\]

\end{itemize}
\end{Ob}

%

%TCIMACRO{\TeXButton{Transition: Box Out}{\transboxout}}%
%BeginExpansion
\transboxout
%EndExpansion%
%TCIMACRO{\TeXButton{EndFrame}{\end{frame}}}%
%BeginExpansion
\end{frame}%
%EndExpansion


\bigskip%

%TCIMACRO{\TeXButton{BeginFrame}{\begin{frame}}}%
%BeginExpansion
\begin{frame}%
%EndExpansion


\bigskip%
%TCIMACRO{\QTR{frametitle}{Base}}%
%BeginExpansion
\frametitle{Base}%
%EndExpansion


\begin{Defi}
\bigskip Sean $M~$un $R$-m\'{o}dulo a derecha y $X\subseteq M$.

\begin{itemize}
\item $X$ es un \textbf{conjunto de generadores }de $M$ si y solo si $|X)=M$.

\item Un m\'{o}dulo est\'{a} \textbf{finitamente generado} si y solo si existe
un conjunto finito de generadores.

\item Un m\'{o}dulo se denomina \textbf{c\'{\i}clico }si y solo si es generado
por un conjunto con un \'{u}nico elemento.

\item $X$ se dice \textbf{libre }si y solo si para todo subconjunto finito de
$X$, $\{x_{1},\ldots,x_{m}\}\subset X$, con $x_{i}\neq x_{j}$ para $i\neq j,$
$i,j=1,\ldots,m$, se cumple que%

\[%
%TCIMACRO{\dsum \limits_{i=1}^{m}}%
%BeginExpansion
{\displaystyle\sum\limits_{i=1}^{m}}
%EndExpansion
x_{i}r_{i}=0\text{, }r_{i}\in R\Rightarrow r_{i}=0,\text{ }\forall
i=1,\ldots,m\text{.}%
\]

\end{itemize}
\end{Defi}

\bigskip

\bigskip%

%TCIMACRO{\TeXButton{Transition: Box Out}{\transboxout}}%
%BeginExpansion
\transboxout
%EndExpansion%
%TCIMACRO{\TeXButton{EndFrame}{\end{frame}}}%
%BeginExpansion
\end{frame}%
%EndExpansion


\bigskip%

%TCIMACRO{\TeXButton{BeginFrame}{\begin{frame}}}%
%BeginExpansion
\begin{frame}%
%EndExpansion


\bigskip%
%TCIMACRO{\QTR{frametitle}{Base}}%
%BeginExpansion
\frametitle{Base}%
%EndExpansion


\begin{Defi}
\bigskip\ $X$ es una \textbf{base }de $M$ si y solo si $X$ es un conjunto
libre de generadores de $M$.
\end{Defi}

Si $X$ es un conjunto de generdaores de $M$, entonces podemos escribir a cada
elemento de $X$ como una combinacion lineal finita de elementos de $X$, pero
esto no significa que la combinaci\'{o}n lineal sea \'{u}nica. Veamos que la
combinaci\'{o}n es \'{u}nica si tenemos una base.

\begin{Lema}
Sea $X\neq\varnothing$ un conjunto generador de $M=M_{R}$. Luego $X$ es una
base si y solo para todo $m\in M$ la representaci\'{o}n%

\[
m=%
%TCIMACRO{\dsum \limits_{i=1}^{n}}%
%BeginExpansion
{\displaystyle\sum\limits_{i=1}^{n}}
%EndExpansion
x_{i}r_{i}\text{, con }x_{i}\in X\text{, }r_{i}\in R\text{ es \'{u}nica.}%
\]

\end{Lema}

\bigskip%

%TCIMACRO{\TeXButton{Transition: Box Out}{\transboxout}}%
%BeginExpansion
\transboxout
%EndExpansion%
%TCIMACRO{\TeXButton{EndFrame}{\end{frame}}}%
%BeginExpansion
\end{frame}%
%EndExpansion


\bigskip%

%TCIMACRO{\TeXButton{BeginFrame}{\begin{frame}}}%
%BeginExpansion
\begin{frame}%
%EndExpansion


\bigskip\qquad%
%TCIMACRO{\QTR{frametitle}{Base}}%
%BeginExpansion
\frametitle{Base}%
%EndExpansion


\textbf{\bigskip Dem}

$\Rightarrow)$ Sea $X$ una base de $M$ y sea $m\in M$, como $X$ es base $m=%
%TCIMACRO{\dsum \limits_{i=1}^{n}}%
%BeginExpansion
{\displaystyle\sum\limits_{i=1}^{n}}
%EndExpansion
x_{i}r_{i}$, con $x_{i}\in X$, $r_{i}\in R.$ Supongamos que $m=%
%TCIMACRO{\dsum \limits_{i=1}^{n}}%
%BeginExpansion
{\displaystyle\sum\limits_{i=1}^{n}}
%EndExpansion
x_{i}r_{i}^{\prime},$ luego

\bigskip$%
%TCIMACRO{\dsum \limits_{i=1}^{n}}%
%BeginExpansion
{\displaystyle\sum\limits_{i=1}^{n}}
%EndExpansion
x_{i}r_{i}=%
%TCIMACRO{\dsum \limits_{i=1}^{n}}%
%BeginExpansion
{\displaystyle\sum\limits_{i=1}^{n}}
%EndExpansion
x_{i}r_{i}^{\prime}\Rightarrow%
%TCIMACRO{\dsum \limits_{i=1}^{n}}%
%BeginExpansion
{\displaystyle\sum\limits_{i=1}^{n}}
%EndExpansion
x_{i}r_{i}-%
%TCIMACRO{\dsum \limits_{i=1}^{n}}%
%BeginExpansion
{\displaystyle\sum\limits_{i=1}^{n}}
%EndExpansion
x_{i}r_{i}^{\prime}=0\Rightarrow%
%TCIMACRO{\dsum \limits_{i=1}^{n}}%
%BeginExpansion
{\displaystyle\sum\limits_{i=1}^{n}}
%EndExpansion
x_{i}(r_{i}-r_{i}^{\prime})=0,$

\bigskip

como $X$ es libre, por ser base, $r_{i}-r_{i}^{\prime}=0$ $\forall
i=1,\ldots,m,$ es decir $r_{i}=r_{i}^{\prime}$ $\forall i=1,\ldots,n.$%

%TCIMACRO{\TeXButton{Transition: Box Out}{\transboxout}}%
%BeginExpansion
\transboxout
%EndExpansion%
%TCIMACRO{\TeXButton{EndFrame}{\end{frame}}}%
%BeginExpansion
\end{frame}%
%EndExpansion


\bigskip%

%TCIMACRO{\TeXButton{BeginFrame}{\begin{frame}}}%
%BeginExpansion
\begin{frame}%
%EndExpansion


\bigskip%
%TCIMACRO{\QTR{frametitle}{Base}}%
%BeginExpansion
\frametitle{Base}%
%EndExpansion


$\Leftarrow)$ Supongamos que para todo $m\in M$ la representaci\'{o}n $m=%
%TCIMACRO{\dsum \limits_{i=1}^{n}}%
%BeginExpansion
{\displaystyle\sum\limits_{i=1}^{n}}
%EndExpansion
x_{i}r_{i}$, con $x_{i}\in X$, $r_{i}\in R$ es \'{u}nica, as\'{\i} claramente
X es un conjunto de generadores de $M,$ veamos que $X$ es libre.

Si $%
%TCIMACRO{\dsum \limits_{i=1}^{n}}%
%BeginExpansion
{\displaystyle\sum\limits_{i=1}^{n}}
%EndExpansion
x_{i}r_{i}=0,$ entonces$%
%TCIMACRO{\dsum \limits_{i=1}^{n}}%
%BeginExpansion
{\displaystyle\sum\limits_{i=1}^{n}}
%EndExpansion
x_{i}r_{i}=%
%TCIMACRO{\dsum \limits_{i=1}^{n}}%
%BeginExpansion
{\displaystyle\sum\limits_{i=1}^{n}}
%EndExpansion
x_{i}0$, as\'{\i} $r_{i}=0$ $\forall i=1,\ldots,n$.

\begin{Ob}
\bigskip Si $X$ es un conjunto de generadores infinito no podemos afirmar la
unicidad de la combinaci\'{o}n pues la suma infinita no est\'{a} definida.
\end{Ob}

\bigskip%

%TCIMACRO{\TeXButton{Transition: Box Out}{\transboxout}}%
%BeginExpansion
\transboxout
%EndExpansion%
%TCIMACRO{\TeXButton{EndFrame}{\end{frame}}}%
%BeginExpansion
\end{frame}%
%EndExpansion


\bigskip

\bigskip%
%TCIMACRO{\TeXButton{BeginFrame}{\begin{frame}}}%
%BeginExpansion
\begin{frame}%
%EndExpansion


\bigskip%
%TCIMACRO{\QTR{frametitle}{Base}}%
%BeginExpansion
\frametitle{Base}%
%EndExpansion


\begin{Ej}
\bigskip

\begin{itemize}
\item Todo m\'{o}dulo $M$ tiene al propio $M$ como conjunto generador, pues
para cada $m\in M$ tenemos la combinaci\'{o}n lineal finita $1=m1,$ $1\in R$.

\item Si $R$ es un anillo, entonces $\{1\}$ es una base de $R_{R}$ y de
$_{R}R$.
\end{itemize}
\end{Ej}

\bigskip

\begin{Propo}
Si eliminamos un n\'{u}mero finito arbitrario de elementos de un conjunto
generador $X$ de $%
%TCIMACRO{\U{211a} }%
%BeginExpansion
\mathbb{Q}
%EndExpansion
_{%
%TCIMACRO{\U{2124} }%
%BeginExpansion
\mathbb{Z}
%EndExpansion
},$ entonces el nuevo conjunto con estos elemntos eliminados es un generador
de $%
%TCIMACRO{\U{211a} }%
%BeginExpansion
\mathbb{Q}
%EndExpansion
_{%
%TCIMACRO{\U{2124} }%
%BeginExpansion
\mathbb{Z}
%EndExpansion
}$.
\end{Propo}

%

%TCIMACRO{\TeXButton{Transition: Box Out}{\transboxout}}%
%BeginExpansion
\transboxout
%EndExpansion%
%TCIMACRO{\TeXButton{EndFrame}{\end{frame}}}%
%BeginExpansion
\end{frame}%
%EndExpansion


\bigskip

\bigskip

%

%TCIMACRO{\TeXButton{BeginFrame}{\begin{frame}}}%
%BeginExpansion
\begin{frame}%
%EndExpansion


\bigskip%
%TCIMACRO{\QTR{frametitle}{Base}}%
%BeginExpansion
\frametitle{Base}%
%EndExpansion


\begin{Ob}
\bigskip La proposici\'{o}n anterior nos dice que $%
%TCIMACRO{\U{211a} }%
%BeginExpansion
\mathbb{Q}
%EndExpansion
_{%
%TCIMACRO{\U{2124} }%
%BeginExpansion
\mathbb{Z}
%EndExpansion
}$ no tiene un conjunto de generadores finito, porque si lo tuviese $%
%TCIMACRO{\U{211a} }%
%BeginExpansion
\mathbb{Q}
%EndExpansion
_{%
%TCIMACRO{\U{2124} }%
%BeginExpansion
\mathbb{Z}
%EndExpansion
}$ seria generado por el conjunto vacio es decir $%
%TCIMACRO{\U{211a} }%
%BeginExpansion
\mathbb{Q}
%EndExpansion
_{%
%TCIMACRO{\U{2124} }%
%BeginExpansion
\mathbb{Z}
%EndExpansion
}=0$.
\end{Ob}

\begin{lemma}
[Lema de Zorn]Sea $A$ un conjunto ordenado. Si todo subconjunto totalmente
ordenado de $A$ tiene una cota superior en $A$ entonces $A$ posee un elemnto maximal.
\end{lemma}

\begin{Propo}
Todo espacio vectorial sobre un cuerpo $K$ tiene base.
\end{Propo}

\bigskip%

%TCIMACRO{\TeXButton{Transition: Box Out}{\transboxout}}%
%BeginExpansion
\transboxout
%EndExpansion%
%TCIMACRO{\TeXButton{EndFrame}{\end{frame}}}%
%BeginExpansion
\end{frame}%
%EndExpansion


\bigskip%

%TCIMACRO{\TeXButton{BeginFrame}{\begin{frame}}}%
%BeginExpansion
\begin{frame}%
%EndExpansion


\bigskip%
%TCIMACRO{\QTR{frametitle}{Suma de m\'{o}dulos}}%
%BeginExpansion
\frametitle{Suma de m\'{o}dulos}%
%EndExpansion


\begin{Propo}
Sea $\Lambda=\{A_{i}$ $/$ $i\in I\}$ un conjunto de subm\'{o}dulos de $M$,
$A_{i}\hookrightarrow M_{R}$. Entonces

$|%
%TCIMACRO{\dbigcup \limits_{i\in I}}%
%BeginExpansion
{\displaystyle\bigcup\limits_{i\in I}}
%EndExpansion
A_{i})=\left\{
\begin{array}
[c]{l}%
\{%
%TCIMACRO{\dsum \limits_{i\in I^{\prime}}}%
%BeginExpansion
{\displaystyle\sum\limits_{i\in I^{\prime}}}
%EndExpansion
a_{i}\text{ }/\text{ }a_{i}\in A_{i}\text{ }\wedge\text{ }I^{\prime}\subset
I\text{ }\wedge\text{ }I^{\prime}\text{ es finito}\}\text{ si }\Lambda
\neq\varnothing\\
\{0\}\text{ si }\Lambda=\varnothing
\end{array}
\right.  $
\end{Propo}

\textbf{Dem. }Sea $\Lambda\neq\varnothing$ y $m\in|%
%TCIMACRO{\dbigcup \limits_{i\in I}}%
%BeginExpansion
{\displaystyle\bigcup\limits_{i\in I}}
%EndExpansion
A_{i})$, entonces por definici\'{o}n $m=%
%TCIMACRO{\dsum \limits_{j=1}^{n}}%
%BeginExpansion
{\displaystyle\sum\limits_{j=1}^{n}}
%EndExpansion
a_{j}r_{j}$, con $a_{j}\in%
%TCIMACRO{\dbigcup \limits_{i\in I}}%
%BeginExpansion
{\displaystyle\bigcup\limits_{i\in I}}
%EndExpansion
A_{i}$, $r_{j}\in R$.

\bigskip Luego para alg\'{u}n $i_{0}$, $a_{j}\in A_{i_{0}}$ y $A_{i}%
\hookrightarrow M_{R},$ para todo $i\in I$ entonces $a_{j}r_{j}=a_{j}^{\prime
}\in A_{i_{0}}.$ As\'{\i}

$|%
%TCIMACRO{\dbigcup \limits_{i}}%
%BeginExpansion
{\displaystyle\bigcup\limits_{i}}
%EndExpansion
A_{i})\hookrightarrow\{%
%TCIMACRO{\dsum \limits_{i}}%
%BeginExpansion
{\displaystyle\sum\limits_{i}}
%EndExpansion
a_{i}$ $/$ $a_{i}\in A_{i}\wedge I^{\prime}\subset I\wedge I^{\prime}$ es
finito$\}.$

\bigskip%

%TCIMACRO{\TeXButton{Transition: Box Out}{\transboxout}}%
%BeginExpansion
\transboxout
%EndExpansion%
%TCIMACRO{\TeXButton{EndFrame}{\end{frame}}}%
%BeginExpansion
\end{frame}%
%EndExpansion


\bigskip%

%TCIMACRO{\TeXButton{BeginFrame}{\begin{frame}}}%
%BeginExpansion
\begin{frame}%
%EndExpansion


\bigskip%
%TCIMACRO{\QTR{frametitle}{Suma de m\'{o}dulos}}%
%BeginExpansion
\frametitle{Suma de m\'{o}dulos}%
%EndExpansion


\bigskip Es claro que $|%
%TCIMACRO{\dbigcup \limits_{i}}%
%BeginExpansion
{\displaystyle\bigcup\limits_{i}}
%EndExpansion
A_{i})\hookleftarrow\{%
%TCIMACRO{\dsum \limits_{i}}%
%BeginExpansion
{\displaystyle\sum\limits_{i}}
%EndExpansion
a_{i}$ $/$ $a_{i}\in A_{i}\wedge I^{\prime}\subset I\wedge I^{\prime}$ es
finito$\}..$

\begin{Defi}
Sea $\Lambda=\{A_{i}$ $/$ $i\in I\}$ un conjunto de subm\'{o}dulos de $M$,
$A_{i}\hookrightarrow M_{R},$ luego%

\[%
%TCIMACRO{\dsum \limits_{i\in I}}%
%BeginExpansion
{\displaystyle\sum\limits_{i\in I}}
%EndExpansion
A_{i}=|%
%TCIMACRO{\dbigcup \limits_{i\in I}}%
%BeginExpansion
{\displaystyle\bigcup\limits_{i\in I}}
%EndExpansion
A_{i})
\]


es llamada \textbf{suma de subm\'{o}dulos }$\{A_{i}$ $/$ $i\in I\}$.
\end{Defi}

%

%TCIMACRO{\TeXButton{Transition: Box Out}{\transboxout}}%
%BeginExpansion
\transboxout
%EndExpansion%
%TCIMACRO{\TeXButton{EndFrame}{\end{frame}}}%
%BeginExpansion
\end{frame}%
%EndExpansion


\bigskip%

%TCIMACRO{\TeXButton{BeginFrame}{\begin{frame}}}%
%BeginExpansion
\begin{frame}%
%EndExpansion


\bigskip%
%TCIMACRO{\QTR{frametitle}{Suma de m\'{o}dulos}}%
%BeginExpansion
\frametitle{Suma de m\'{o}dulos}%
%EndExpansion


\begin{Ob}
\bigskip

\begin{itemize}
\item Si $\Lambda=\{A_{1},\ldots,A_{n}\}$, notamos $%
%TCIMACRO{\dsum \limits_{i=1}^{n}}%
%BeginExpansion
{\displaystyle\sum\limits_{i=1}^{n}}
%EndExpansion
A_{i}$ y sus elementos son de la forma $%
%TCIMACRO{\dsum \limits_{i=1}^{n}}%
%BeginExpansion
{\displaystyle\sum\limits_{i=1}^{n}}
%EndExpansion
a_{i}$ con $a_{i}\in A_{i}$.

\item No podemos afirmar que la representacion de los elementos de la suma de
subm\'{o}dulos es \'{u}nica.
\end{itemize}
\end{Ob}

%

%TCIMACRO{\TeXButton{Transition: Box Out}{\transboxout}}%
%BeginExpansion
\transboxout
%EndExpansion%
%TCIMACRO{\TeXButton{EndFrame}{\end{frame}}}%
%BeginExpansion
\end{frame}%
%EndExpansion


\bigskip%

%TCIMACRO{\TeXButton{BeginFrame}{\begin{frame}}}%
%BeginExpansion
\begin{frame}%
%EndExpansion


\bigskip%
%TCIMACRO{\QTR{frametitle}{Suma de m\'{o}dulos}}%
%BeginExpansion
\frametitle{Suma de m\'{o}dulos}%
%EndExpansion


\begin{Lema}
\bigskip Sea $A\hookrightarrow_{\neq}M,$ luego es equivalente

\begin{enumerate}
\item $A$ es subm\'{o}dulo m\'{a}ximal de $M.$

\item $\forall m\in M$ $:$ $m\notin A\Rightarrow M=mR+A$.
\end{enumerate}
\end{Lema}

\bigskip\textbf{Dem. }(1)$\Rightarrow$(2). Sea $m\notin A$, luego
$A\hookrightarrow_{\neq}mR+A$ y como $A$ es subm\'{o}dulo principal $mR+A=M$.

(2)$\Rightarrow$(1). Considerems $A\hookrightarrow_{\neq}B\hookrightarrow M$ y
sea $m\in B$ tal que $m\notin A.$

\bigskip

Como $m\in B$ $mR\hookrightarrow B,$ luego $mR+A\hookrightarrow B+A,$ as\'{\i}

%

\[
M=mR+A\hookrightarrow B+A\hookrightarrow B\hookrightarrow M.
\]


\bigskip

Por lo tanto $M=B$.

\bigskip%

%TCIMACRO{\TeXButton{Transition: Box Out}{\transboxout}}%
%BeginExpansion
\transboxout
%EndExpansion%
%TCIMACRO{\TeXButton{EndFrame}{\end{frame}}}%
%BeginExpansion
\end{frame}%
%EndExpansion


\bigskip%

%TCIMACRO{\TeXButton{BeginFrame}{\begin{frame}}}%
%BeginExpansion
\begin{frame}%
%EndExpansion


\bigskip%
%TCIMACRO{\QTR{frametitle}{Suma de m\'{o}dulos}}%
%BeginExpansion
\frametitle{Suma de m\'{o}dulos}%
%EndExpansion


\begin{Teo}
Si el m\'{o}dulo $M_{R}$ esta finitamente generado entonces cada subm\'{o}dulo
propio de $M$ esta contenido en un subm\'{o}dulo maximal.
\end{Teo}

\qquad

\textbf{Dem. }Sea $\{m_{1},\ldots,m_{t}\}$ un conjunto de generadores de $M$ y
sea $A\hookrightarrow_{\neq}M,$ luego el conjunto%

\[
\Phi=\{B/A\hookrightarrow B\bigskip\hookrightarrow_{\neq}M\}
\]


es distinto de vacio pues $A\in\Phi.$

\bigskip%

%TCIMACRO{\TeXButton{Transition: Box Out}{\transboxout}}%
%BeginExpansion
\transboxout
%EndExpansion%
%TCIMACRO{\TeXButton{EndFrame}{\end{frame}}}%
%BeginExpansion
\end{frame}%
%EndExpansion


\bigskip%

%TCIMACRO{\TeXButton{BeginFrame}{\begin{frame}}}%
%BeginExpansion
\begin{frame}%
%EndExpansion


\bigskip%
%TCIMACRO{\QTR{frametitle}{Suma de m\'{o}dulos}}%
%BeginExpansion
\frametitle{Suma de m\'{o}dulos}%
%EndExpansion


\bigskip Buscamos aplicar el Lema de Zorn, para ello debemos ver que todo
subconjunto totalmente ordenado tiene cota superior.

Sea $\Gamma\subseteq\Phi$ un subconjunto totalmente ordenado y sea $C=%
%TCIMACRO{\dbigcup \limits_{B\in\Gamma}}%
%BeginExpansion
{\displaystyle\bigcup\limits_{B\in\Gamma}}
%EndExpansion
B$ notamos que $C$ verifica que $A\hookrightarrow C.$

Si suponemos que $C=M$ entonces $\{m_{1},\ldots,m_{t}\}\subseteq C$ y existe
$B\in\Gamma$ tal que $\{m_{1},\ldots,m_{t}\}\subseteq B,$ luego $B=M,$ absurdo.

Luego $C\in\Phi$ y por el Lema de Zorn existe un elemento maximal $D$ en
$\Phi.$ Veamos que $D$ es un subm\'{o}dulo maximal de $M_{R},$ sea
$D\hookrightarrow L\hookrightarrow_{\neq}M,$ entonces $L\in\Phi$ y como $D$ es
elemento maximal $D=L$.%

%TCIMACRO{\TeXButton{Transition: Box Out}{\transboxout}}%
%BeginExpansion
\transboxout
%EndExpansion%
%TCIMACRO{\TeXButton{EndFrame}{\end{frame}}}%
%BeginExpansion
\end{frame}%
%EndExpansion


\bigskip%

%TCIMACRO{\TeXButton{BeginFrame}{\begin{frame}}}%
%BeginExpansion
\begin{frame}%
%EndExpansion


\bigskip%
%TCIMACRO{\QTR{frametitle}{Suma de m\'{o}dulos}}%
%BeginExpansion
\frametitle{Suma de m\'{o}dulos}%
%EndExpansion


\begin{Corol}
Todo m\'{o}dulo $M\neq\varnothing$ finitamente generado tiene un subm\'{o}dulo maximal.
\end{Corol}

\begin{Teo}
El m\'{o}dulo $M_{R}$ esta finitamente generado si y solo si en todo conjunto
$\{A_{i}/i\in I,A_{i}\hookrightarrow M\}$ que verifica $%
%TCIMACRO{\dsum \limits_{i\in I}}%
%BeginExpansion
{\displaystyle\sum\limits_{i\in I}}
%EndExpansion
A_{i}=M$ existe un subconjunto finito $\{A_{i}/i\in I_{\text{\thinspace}%
0},I_{0}\subseteq I$ finito$\}$ tal que%

\[%
%TCIMACRO{\dsum \limits_{i\in I_{0}}}%
%BeginExpansion
{\displaystyle\sum\limits_{i\in I_{0}}}
%EndExpansion
A_{i}=M
\]

\end{Teo}

%

%TCIMACRO{\TeXButton{Transition: Box Out}{\transboxout}}%
%BeginExpansion
\transboxout
%EndExpansion%
%TCIMACRO{\TeXButton{EndFrame}{\end{frame}}}%
%BeginExpansion
\end{frame}%
%EndExpansion


\bigskip%

%TCIMACRO{\TeXButton{BeginFrame}{\begin{frame}}}%
%BeginExpansion
\begin{frame}%
%EndExpansion


\bigskip%
%TCIMACRO{\QTR{frametitle}{Suma de m\'{o}dulos}}%
%BeginExpansion
\frametitle{Suma de m\'{o}dulos}%
%EndExpansion


\bigskip\textbf{Dem. }$\Rightarrow).$ Sea $M$ finitamente generado, entonces
existe un conjunto finito de generadores, $\{m_{1},\ldots,m_{t}\}$, as\'{\i}
$M=m_{1}R+\ldots+m_{t}R,$ si el conjunto $\{A_{i}$ $/$ $i\in I,$
$A_{i}\hookrightarrow M\}$ verifica $%
%TCIMACRO{\dsum \limits_{i\in I}}%
%BeginExpansion
{\displaystyle\sum\limits_{i\in I}}
%EndExpansion
A_{i}=M$ , entonces cada $m_{i}$ es una suma finita de elementos de $A_{i}.$
As\'{\i} existe $I_{0}\subset I$ finito tal que $m_{1},\ldots,m_{t}\in%
%TCIMACRO{\dsum \limits_{i\in I_{0}}}%
%BeginExpansion
{\displaystyle\sum\limits_{i\in I_{0}}}
%EndExpansion
A_{i}.$ Sigue que $M=m_{1}R+\ldots+m_{t}R\hookrightarrow%
%TCIMACRO{\dsum \limits_{i\in I_{0}}}%
%BeginExpansion
{\displaystyle\sum\limits_{i\in I_{0}}}
%EndExpansion
A_{i}\hookrightarrow M$. Por lo tanto $%
%TCIMACRO{\dsum \limits_{i\in I_{0}}}%
%BeginExpansion
{\displaystyle\sum\limits_{i\in I_{0}}}
%EndExpansion
A_{i}=M.$

$\Longleftarrow).$ Consideramos el conjunto $\{mR$ $/$ $m\in M\}$ que verifica
$%
%TCIMACRO{\dsum \limits_{m\in M}}%
%BeginExpansion
{\displaystyle\sum\limits_{m\in M}}
%EndExpansion
mR=M,$ por hip\'{o}tesis existe un conjunto finito $\{m_{1}R,\ldots,m_{t}R\}$
tal que $m_{1}R+\ldots+m_{t}R=M.$ Por lo tanto $M$ esta finitamente generado.

\bigskip

\bigskip%

%TCIMACRO{\TeXButton{Transition: Box Out}{\transboxout}}%
%BeginExpansion
\transboxout
%EndExpansion%
%TCIMACRO{\TeXButton{EndFrame}{\end{frame}}}%
%BeginExpansion
\end{frame}%
%EndExpansion


\bigskip%

%TCIMACRO{\TeXButton{BeginFrame}{\begin{frame}}}%
%BeginExpansion
\begin{frame}%
%EndExpansion


\bigskip%
%TCIMACRO{\QTR{frametitle}{Suma de m\'{o}dulos}}%
%BeginExpansion
\frametitle{Suma de m\'{o}dulos}%
%EndExpansion


\begin{Defi}
Diremos que $M_{R}$ es \textbf{finitamente congregado }si y solo si para cada
conjunto $\{A_{i}$ $/$ $i\in I,$ $A_{i}\hookrightarrow M\}$ que verifca $%
%TCIMACRO{\dbigcap \limits_{i\in I}}%
%BeginExpansion
{\displaystyle\bigcap\limits_{i\in I}}
%EndExpansion
A_{i}=\{0\}$ existe un subconjunto finito $\{A_{i}$ $/$ $i\in
I_{\text{\thinspace}0},$ $I_{0}\subseteq I$ finito$\}$ tal que $%
%TCIMACRO{\dbigcap \limits_{i\in I_{0}}}%
%BeginExpansion
{\displaystyle\bigcap\limits_{i\in I_{0}}}
%EndExpansion
A_{i}=\{0\}.$
\end{Defi}

\begin{lemma}
[Ley Modular]Sean $A,$ $B$, $C\hookrightarrow M$ y $B\hookrightarrow C$ entonces%

\[
(A+B)\cap C=(A\cap C)+(B\cap C)=(A\cap C)+B\text{.}%
\]

\end{lemma}

\bigskip%

%TCIMACRO{\TeXButton{Transition: Box Out}{\transboxout}}%
%BeginExpansion
\transboxout
%EndExpansion%
%TCIMACRO{\TeXButton{EndFrame}{\end{frame}}}%
%BeginExpansion
\end{frame}%
%EndExpansion


\bigskip%

%TCIMACRO{\TeXButton{BeginFrame}{\begin{frame}}}%
%BeginExpansion
\begin{frame}%
%EndExpansion


\bigskip%
%TCIMACRO{\QTR{frametitle}{Suma de m\'{o}dulos}}%
%BeginExpansion
\frametitle{Suma de m\'{o}dulos}%
%EndExpansion


\bigskip\textbf{Dem}. Sean $a+b\in(A+B)\cap C,$ entonces $a+b=c\in C,$
as\'{\i} $a=c-b$ y como $B\hookrightarrow C,$ $a=c-b\in A\cap C,$ entonces
$a+b=c\in(A\cap C)+B$. Por lo tanto $(A+B)\cap C\hookrightarrow(A\cap C)+B$.

Sea $d\in(A\cap C)$ y $b\in B.$ Como $B\hookrightarrow C$ tenmos que
$d+b\in(A+B)\cap C.$ Por lo tanto, $(A\cap C)+B\hookrightarrow(A+B)\cap C$

\bigskip

\bigskip%

%TCIMACRO{\TeXButton{Transition: Box Out}{\transboxout}}%
%BeginExpansion
\transboxout
%EndExpansion%
%TCIMACRO{\TeXButton{EndFrame}{\end{frame}}}%
%BeginExpansion
\end{frame}%
%EndExpansion



\end{document}