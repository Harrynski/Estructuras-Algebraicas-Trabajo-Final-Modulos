\documentclass[notes=show]{beamer}%
\usepackage{mathpazo}
\usepackage{hyperref}
\usepackage{multimedia}
\usepackage{amsmath}
\usepackage{amsfonts}
\usepackage{amssymb}
\usepackage{graphicx}%
\setcounter{MaxMatrixCols}{30}
%TCIDATA{OutputFilter=latex2.dll}
%TCIDATA{Version=5.50.0.2953}
%TCIDATA{CSTFile=beamer.cst}
%TCIDATA{Created=Thursday, July 01, 2021 12:41:31}
%TCIDATA{LastRevised=Sunday, July 31, 2022 20:20:43}
%TCIDATA{<META NAME="GraphicsSave" CONTENT="32">}
%TCIDATA{<META NAME="SaveForMode" CONTENT="1">}
%TCIDATA{BibliographyScheme=Manual}
%TCIDATA{<META NAME="DocumentShell" CONTENT="Other Documents\SW\Slides - Beamer">}
%TCIDATA{Language=American English}
%BeginMSIPreambleData
\providecommand{\U}[1]{\protect\rule{.1in}{.1in}}
%EndMSIPreambleData
\newenvironment{Defi}{\begin{block}{Definición}}{\end{block}}
\newenvironment{Lema}{\begin{block}{Lema}}{\end{block}}
\newenvironment{Corol}{\begin{block}{Corolario}}{\end{block}}
\newenvironment{Propo}{\begin{block}{Proposición}}{\end{block}}
\newenvironment{Teo}{\begin{block}{Teorema}}{\end{block}}
\newenvironment{Ej}{\begin{block}{Ejemplo}}{\end{block}}
\newenvironment{Ob}{\begin{block}{Observación}}{\end{block}}
\newenvironment{stepenumerate}{\begin{enumerate}[<+->]}{\end{enumerate}}
\newenvironment{stepitemize}{\begin{itemize}[<+->]}{\end{itemize} }
\newenvironment{stepenumeratewithalert}{\begin{enumerate}[<+-| alert@+>]}{\end{enumerate}}
\newenvironment{stepitemizewithalert}{\begin{itemize}[<+-| alert@+>]}{\end{itemize} }
\newenvironment{ObV}{\begin{exampleblock}{Observación}}{\end{exampleblock}}
\newenvironment{PropoR}{\begin{alertblock}{Proposición}}{\end{alertblock}}
\newenvironment{EjV}{\begin{exampleblock}{Ejemplo}}{\end{exampleblock}}
\usetheme{CambridgeUS}
\usecolortheme{seahorse}
\begin{document}

\title{Trabajo Final: Módulos}
\subtitle{Estructuras Algebraicas}
\author{Nicolas Silva Nash}
\institute{UNCO}
\date{}
\maketitle

\bigskip%
\begin{frame}
    \frametitle{Definición de Módulo}
    Sea \( R \) un anillo. Un \( R \)-módulo por derecha \( M \) es
    \begin{enumerate}
        \item[(I)] un grupo abeliano aditivo $(M, +)$ junto con
        \item[(II)] una aplicación
        \[
        M \times R \;\rightarrow\; M \quad \text{con} \quad (m, r) \mapsto mr,
        \]
        llamada \textit{multiplicación de módulo}, para la cual tenemos
        \begin{enumerate}
            \item Ley asociativa: \((mr_1)r_2 = m(r_1r_2)\).
            \item Leyes distributivas: 
            \[
            (m_1 + m_2)r = m_1r + m_2r, \quad m(r_1 + r_2) = mr_1 + mr_2.
            \]
            \item Ley unitaria: \( m \cdot 1 = m \).
        \end{enumerate}
    \end{enumerate}
    Con \( m, m_1, m_2 \) elementos arbitrarios de $M$ y \( r, r_1, r_2 \) elementos arbitrarios de $R$.
\end{frame}

% FRAME 2
\begin{frame}
    \frametitle{Módulos}
    Algunas observaciones:
    \begin{enumerate}
        \item Podemos definir de manera análoga un \( R \)-módulo por izquierda.
        \item Notamos \( M_R \) al \( R \)-módulo por derecha y \( _{R}M \) al \( R \)-módulo por izquierda.
        \item Si un módulo verifica ambas condiciones para anillos \( R \) (por derecha) y \( S \) (por izquierda), 
        y además verifica: 
        \[ s(mr) = (sm)r, \qquad \forall s \in S, r \in R \] 
        decimos que es un \( S-R \)-bimódulo al que notamos \( _{S}M_R \).
        \item Si \( 0_M \) es el cero de \( M \), \( 0_R \) es el cero de \( R \), entonces:
        \begin{itemize}
            \item \( 0_M \cdot r = 0_M \)
            \item \( m \cdot 0_R = 0_M \), \(\forall m \in M, r \in R \)
        \end{itemize}
    \end{enumerate}
\end{frame}

% FRAME 3
\begin{frame}
    \frametitle{Submódulos}
    \textbf{Definición}: 
    Sea \( M \) un \( R \)-módulo por derecha. Un subconjunto \( A \) de \( M \) se llama un submódulo de \( M \), 
    notacionalmente \( A \hookrightarrow M \) (o también \( A_R \hookrightarrow M_R \)) si \( A \) es un \( R \)-módulo 
    por derecha con respecto a la restricción de la suma y la multiplicación de módulo de \( M \) a \( A \).\\
    \bigskip
    Usamos la notación \( A \hookrightarrow M \) para la relación de submódulo, para tener disponible \( A \subseteq M \) 
    para la inclusión en teoría de conjuntos. Además, denotamos \( A \hookrightarrow_{\neq} M \) si y sólo si \( A \) 
    es un submódulo propio de \( M \).\\
    \bigskip
    Notamos \( A \not\hookrightarrow M \) si \( A \) no es un submódulo de \( M \). Observamos que de \( A \not\hookrightarrow M \) 
    no necesariamente se sigue que \( A \nsubseteq M \).
\end{frame}

\begin{frame}
    \frametitle{Submódulos}
    \textbf{Lema}:\\ 
    Sea \( M \) un \( R \)-módulo por derecha. Si \( A \) es un subconjunto de \( M \) y \( A \neq \emptyset \), entonces las 
    siguientes afirmaciones son equivalentes:
    \begin{enumerate}
        \item \( A \hookrightarrow M \).
        \item \( A \) es un subgrupo del grupo aditivo de \( M \) y para todo \( a \in A \) y todo \( r \in R \), tenemos 
        \( ar \in A \) (donde \( ar \) es la multiplicación de módulo en \( M \)).
        \item Para todos \( a_1, a_2 \in A \), \( a_1 + a_2 \in A \) (con respecto a la suma en \( M \)) y para todo 
        \( a \in A \) y todo \( r \in R \), tenemos \( ar \in A \).
    \end{enumerate}
\end{frame}

\begin{frame}
    \frametitle{Submódulos: Ejemplos y observaciones}
    \begin{enumerate}
        \item Todo módulo \( M \) posee los submódulos triviales \( 0 \) y \( M \), donde \( 0 \) es el submódulo 
        que contiene solo el elemento cero de \( M \).
        \item Sea \( M \) arbitrario y sea \( m_0 \in M \).
        \[
        m_0R = \{m_0r \mid r \in R\}
        \]
        es un submódulo de \( M \) que se llama el submódulo cíclico de \( M \) generado por \( m_0 \).
        \item Si \( M_K \) es un espacio vectorial sobre el campo \( K \), entonces los submódulos se 
        laman subespacios (lineales).
        \item En el anillo \( \mathbb{Z} \) de los números naturales, cada ideal es cíclico.
        \item Los ideales cíclicos de un anillo se llaman ideales principales y un anillo conmutativo se 
        llama anillo de ideales principales si cada ideal es un ideal principal.
        \item Un campo \( K \) tiene solo los ideales triviales \( 0 \) y \( K \).
    \end{enumerate}
\end{frame}

\begin{frame}
    \frametitle{Submódulos: Definiciones}
    \begin{enumerate}
        \item[(1)] Un módulo \( M = M_R \) se llama \textit{cíclico} si y solo si
        \[ \exists m_0 \in M : \; M = m_0R \]
        \item[(2)] Un anillo \( R \) se llama \textit{simple} si y solo si
        \[
        \forall A \hookrightarrow R: \; A = 0 \text{ or} A = R,
        \]
        es decir, \( 0 \) y \( R \) son los únicos ideales bilaterales de \( R \).
        \item[(3)] Un submódulo \( A \hookrightarrow M \) se dice un \textit{submódulo minimal} de \( M \) si y solo si
        \[
        0 \hookrightarrow B \hookrightarrow A \;\Rightarrow\; B = 0 \text{ o } B = A,
        \]
        \item[(4)] Un submódulo \( A \hookrightarrow M \) se dice un \textit{submódulo maximal} sí y solo si
     \[
            A \hookrightarrow B \hookrightarrow M \;\Rightarrow\; B = A \text{ o } B = M.
            \]
    \end{enumerate}
\end{frame}

\begin{frame}
    \frametitle{Submódulos}
    \textbf{Lema.} \( M \) es simple si y solo si
    \begin{enumerate}
        \item \( M \neq 0\)
        \item \( \forall m \in M: \; m \neq 0 \Rightarrow mR = M \)
    \end{enumerate}
    \textbf{Prueba.}
    \begin{itemize}
        \item \textbf{(\( \Rightarrow \))}: 
        Supongamos que \( M \) es simple. Sea \( m \in M, m\neq 0 \). Entonces \( m = m \cdot 1 \in mR \), luego \(mR \neq 0\). 
        Como \(mR\subset M\) y \(M\) es simple, tenemos que necesariamente \( mR = M\). 
        \item \textbf{(\( \Leftarrow \))}: Sea \(A\) tal que \( 0 \hookrightarrow_{\neq} A \hookrightarrow_{\neq} M \) y sea \(a\in A, a\neq 0\).
        Luego \(aR\in A \). Además, por la hipótesis, \(a \in M\) implica \(aR = M\). Sigue que \(aR = M \subset A\), luego \(A = M\).
    \end{itemize}
\end{frame}

\begin{frame}
    \frametitle{Ejemplos}
    \begin{enumerate}
        \item \( \mathbb{Z} \) no contiene ideales minimales (simples), ya que si \( n\mathbb{Z} \neq 0 \), 
        entonces \( 2n\mathbb{Z} \) es un ideal no nulo contenido dentro de \( n\mathbb{Z} \). 
        \item Los ideales maximales de \( \mathbb{Z} \) son exactamente los ideales primos \( p\mathbb{Z} \), donde \( p \) 
        es un número primo. La prueba de esto sigue del hecho de que
        \[
        m\mathbb{Z} \hookrightarrow n\mathbb{Z} \iff n \mid m.
        \]
        \item \( \mathbb{Q}_{\mathbb{Z}} \) no tiene submódulos minimales ni maximales.\\
        Supongamos que \( A \) es un submódulo no trivial de \( \mathbb{Q}_{\mathbb{Z}} \). \\
        Sea \(A\) tal que \( 0 \hookrightarrow_{\neq} A \hookrightarrow \mathbb{Q}_{\mathbb{Z}} \). 
        Sea \( a \in A\) tal que \(a\neq 0\).
        Entonces
        \[
        0 \hookrightarrow_{\neq} 2a\mathbb{Z} \hookrightarrow_{\neq} a\mathbb{Z} \hookrightarrow A \hookrightarrow \mathbb{Q}.
        \]
        Por lo tanto, \( A \) no puede ser minimal. 
        Más adelante veremos que no existen submódulos maximales.
    \end{enumerate}
\end{frame}

\begin{frame}
    \frametitle{Definición de Álgebra}
    Aprovechamos esta oportunidad para recordar la definición de un álgebra.\\
    
    \textbf{Definición.} Un álgebra es un par \( (R, K) \), donde
    \begin{enumerate}
        \item[(I)] \( R \) es un anillo.
        \item[(II)] \( K \) es un anillo conmutativo.
        \item[(III)] \( R \) es un módulo derecho sobre \( K \) para el cual se cumple
        \[
        \forall r_1, r_2 \in R, \, k \in K: \quad (r_1 r_2) k = r_1 (r_2 k) = (r_1 k) r_2.
        \]
    \end{enumerate}


    Hemos definido a \( R \) con un elemento unitario y a \( K \) actuando unitariamente sobre \( R \). 
    El par \( (R, K) \) también se denomina \( K \)-álgebra o álgebra sobre \( K \).\\
    No tiene sentid definir a \( R \) como una "álgebra derecha sobre \( K \)". Dado que \( K \) es conmutativo, podemos, a partir de la definición
    \[
    kr := rk, \quad \forall r \in R, k \in K,
    \]
    pasar inmediatamente a una "álgebra izquierda sobre \( K \)".
    
\end{frame}







\end{document}